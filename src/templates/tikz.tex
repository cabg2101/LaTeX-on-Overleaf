%%%%%%%%%%%%%%%%%%%%%%%%%%%%%%%%%%%%%%%%%%%%%%%%%%
%%% SIMPLE TIKZ EXAMPLES                       %%%
%%%%%%%%%%%%%%%%%%%%%%%%%%%%%%%%%%%%%%%%%%%%%%%%%%

\begin{figure}[H]
    \centering
    \begin{tikzpicture}
        \draw[gray, thick] (-1,2) -- (2,-4);
        \draw[gray, thick] (-1,-1) -- (2,2);
        \filldraw[black] (0,0) circle (2pt) node[anchor=west] {Intersection point};
    \end{tikzpicture}
    \caption[Template: TikZ \#1]{Caption under figure.}
    \label{fig:tikzlabel_01}
\end{figure}

\begin{figure}[H]
    \centering
    \begin{subfigure}[t]{0.5\textwidth}
        \centering
        \begin{tikzpicture}
            \draw (-2,0) -- (2,0);
            \filldraw [gray] (0,0) circle (2pt);
            \draw (-2,-2) .. controls (0,0) .. (2,-2);
            \draw (-2,2) .. controls (-1,0) and (1,0) .. (2,2);
        \end{tikzpicture}
        \caption{Caption under subfigure.}
        \label{fig:tikzsublabel_01}
    \end{subfigure}%
    \begin{subfigure}[t]{0.5\textwidth}
        \centering
        \begin{tikzpicture}
            \draw (-2,0) -- (2,0);
            \filldraw [gray] (0,0) circle (2pt);
            \draw (-2,-2) .. controls (0,0) .. (2,-2);
            \draw (-2,2) .. controls (-1,0) and (1,0) .. (2,2);
        \end{tikzpicture}
        \caption{Caption under subfigure.}
        \label{fig:tikzsublabel_02}
    \end{subfigure}
    \caption[Template: TikZ \#2]{Caption under figure.}
    \label{fig:tikzlabel_02}
\end{figure}

%%%%%%%%%%%%%%%%%%%%%%%%%%%%%%%%%%%%%%%%%%%%%%%%%%
%%% SYMBOLS / GEOMETRY                         %%%
%%%%%%%%%%%%%%%%%%%%%%%%%%%%%%%%%%%%%%%%%%%%%%%%%%

\begin{figure}[H]
    \centering
    \begin{tikzpicture}
        \filldraw[color=red!60, fill=red!5, very thick](-1,0) circle (1.5);
        \fill[blue!50] (2.5,0) ellipse (1.5 and 0.5);
        \draw[ultra thick, ->] (6.5,0) arc (0:220:1);
    \end{tikzpicture}
    \caption[Template: TikZ \#3]{Caption under figure.}
    \label{fig:tikzlabel_03}
\end{figure}

\begin{figure}[H]
    \hspace*{\fill}
    \begin{tikzpicture}
        \draw[blue, very thick] (0,0) rectangle (3,2);
    \end{tikzpicture}
    \hfill
    \begin{tikzpicture}
        \draw[orange, ultra thick] (4,0) -- (6,0) -- (5.7,2) -- cycle;
    \end{tikzpicture}
    \hspace*{\fill}
    \caption[Template: TikZ \#4]{Caption under figure.}
    \label{fig:tikzlabel_04}
\end{figure}

%%%%%%%%%%%%%%%%%%%%%%%%%%%%%%%%%%%%%%%%%%%%%%%%%%
%%% NODE GRAPH                                 %%%
%%%%%%%%%%%%%%%%%%%%%%%%%%%%%%%%%%%%%%%%%%%%%%%%%%

\begin{figure}[H]
    \centering
    \begin{tikzpicture}[
        roundnode/.style={%
            circle,
            draw=green!60,
            fill=green!5,
            very thick,
            minimum size=7mm
        },
        squarednode/.style={%
            rectangle,
            draw=red!60,
            fill=red!5,
            very thick,
            minimum size=5mm
        },
    ]
        % Nodes
        \node[squarednode]  (maintopic)                          {2};
        \node[roundnode]    (uppercircle)   [above=of maintopic] {1};
        \node[squarednode]  (rightsquare)   [right=of maintopic] {3};
        \node[roundnode]    (lowercircle)   [below=of maintopic] {4};
        
        % Lines
        \draw[->] (uppercircle.south) -- (maintopic.north);
        \draw[->] (maintopic.east)    -- (rightsquare.west);
        \draw[->] (rightsquare.south) .. controls +(down:7mm) and +(right:7mm) .. (lowercircle.east);
    \end{tikzpicture}
    \caption[Template: TikZ \#5]{Caption under figure.}
    \label{fig:tikzlabel_05}
\end{figure}

%%%%%%%%%%%%%%%%%%%%%%%%%%%%%%%%%%%%%%%%%%%%%%%%%%
%%% COLORED TABLE                              %%%
%%%%%%%%%%%%%%%%%%%%%%%%%%%%%%%%%%%%%%%%%%%%%%%%%%

% Required librairies:
% - matrix

\definecolor{mygray}{HTML}{C0C0C0}
\definecolor{mylightgray}{HTML}{EFEFEF}
\definecolor{myred}{HTML}{FD6864}
\definecolor{mylightred}{HTML}{FFCCC9}
\definecolor{myyellow}{HTML}{F6DB53}
\definecolor{mylightyellow}{HTML}{FDF0AD}
\definecolor{mygreen}{HTML}{77CA5C}
\definecolor{mylightgreen}{HTML}{CFEAC6}
\definecolor{mypurple}{HTML}{CDA7F6}
\definecolor{mylightpurple}{HTML}{F2E7FD}

\begin{table}[H]
    \small
    
    \setlength{\columnA}{\dimexpr 0.16\linewidth}
    \setlength{\columnB}{\dimexpr 0.60\linewidth}
    \setlength{\columnC}{\dimexpr 0.12\linewidth}
    \setlength{\columnD}{\dimexpr 0.12\linewidth}
    
    \setlength{\columnA}{\columnA-2\tabcolsep-5\vbar/4-3.71312pt}
    \setlength{\columnB}{\columnB-2\tabcolsep-5\vbar/4}
    \setlength{\columnC}{\columnC-2\tabcolsep-5\vbar/4}
    \setlength{\columnD}{\columnD-2\tabcolsep-5\vbar/4}
    
    \caption[Template: TikZ colored table]{Caption over table.}
    \label{tab:tikzlabel_01}
    
    \begin{tikzpicture}[
        cell/.style = {
            rectangle,
            draw = black
        },
        nodes in empty cells
    ]
        \matrix(table)[
            matrix of nodes,
            row sep    =-\pgflinewidth,
            column sep = -\pgflinewidth,
            nodes      = {anchor=base,text height=2ex,text depth=2ex},
            column 1/.style = {nodes={cell, align=center, text width=\columnA, font=\bfseries}},
            column 2/.style = {nodes={cell, align=left, text width=\columnB}},
            column 3/.style = {nodes={cell, align=center, text width=\columnC}},
            column 4/.style = {nodes={cell, align=center, text width=\columnD}},
            row 1/.style = {nodes={cell, text height=1em, text depth=0.25em, fill=mygray}},
            row 2/.style = {nodes={cell, text height=1em, text depth=2em, fill=mylightgray, font=\bfseries}},
            row 3/.style = {nodes={cell, text height=1em, text depth=2em,  fill=mylightred}},
            row 4/.style = {nodes={cell, text height=1em, text depth=2em, fill=mylightyellow}},
            row 5/.style = {nodes={cell, text height=1em, text depth=2em, fill=mygreen}},
            row 6/.style = {nodes={cell, text height=1em, text depth=7em,    fill=mylightgreen}},
            row 7/.style = {nodes={cell, text height=1em, text depth=2em, fill=mylightgreen}},
            row 8/.style = {nodes={cell, text height=1em, text depth=2em, fill=mylightgreen}},
            row 9/.style = {nodes={cell, text height=1em, text depth=0.25em, fill=mylightgreen}},
            row 10/.style= {nodes={cell, text height=1em, text depth=2em, fill=mypurple}},
            row 11/.style= {nodes={cell, text height=1em, text depth=2em, fill=mylightpurple}},
            row 12/.style= {nodes={cell, text height=1em, text depth=2em, fill=mylightpurple}},
            row 13/.style= {nodes={cell, text height=1em, text depth=2em, fill=mylightpurple}},
        ] 
        {
            TMF Tasks
                &
                &
                &
            \\
            Task
                &Description
                &Starting Date
                &End Date
            \\
            |[fill=myred]| Project Start 
                &First meeting at which the project was planned (its structure and the objectives to be achieved at the end of the project). 
                &30/11/18
                &30/11/18
            \\
            |[fill=myyellow]| Theoretical Analysis 
                &Initial study of the system based on the documentation 
                &01/12/18
                &23/12/18
            \\
            Matlab simulations
                &
                &
                &
            \\
            Development of equations 
                &The first equations were developed in Matlab, verifying that the results from the University of Rochester {[}REF{]} were obtained for the cases of 3 and 4 lenses. Later, the system was generalized for the case of 5 lenses, obtaining the first problems. 
                &24/12/18
                &10/01/19
            \\
            Meeting 
                &To comment on the progress made and try to solve  problems in simulations with Matlab. 
                &31/01/19
                &31/01/19
            \\ 
            Meeting 
                &To compare the Matlab equations and try to find  the solution. 
                &06/03/19
                &06/03/19
            \\
            Meeting 
                &Finally, we fixed the problem with the Matlab  equations. 
                &08/04/19
                &08/04/19
            \\
            Zemax simulations
                &
                &
                &
            \\ 
            Zemax installation 
                &We had some problems during the installation of the program due to the Windows version of the computer. 
                &01/01/19
                &03/03/19
            \\ 
            First steps with Zemax 
                &First days learning to use the program. 
                &25/03/19
                &30/03/19
            \\ 
            Meeting 
                &Skype meeting to clarify some doubts regarding Zemax simulations. 
                &27/04/19
                &27/04/19
            \\
        };
    \end{tikzpicture}
\end{table}

%%%%%%%%%%%%%%%%%%%%%%%%%%%%%%%%%%%%%%%%%%%%%%%%%%
%%% MATRICES                                   %%%
%%%%%%%%%%%%%%%%%%%%%%%%%%%%%%%%%%%%%%%%%%%%%%%%%%

% Required librairies:
% - matrix
% - decorations.pathreplacing
% - calc
% - fit 
% - backgrounds

\pgfmathsetmacro{\myscale}{1.25}
\pgfkeys{
    tikz/mymatrixenv/.style = {
        decoration                   = {brace},
        every left delimiter/.style  = {xshift=8pt},
        every right delimiter/.style = {xshift=-8pt}
    }
}
\pgfkeys{
    tikz/mymatrix/.style={
        matrix of math nodes,
        nodes in empty cells,
        left delimiter  = {[},
        right delimiter = {]},
        inner sep       = 1pt,
        outer sep       = 1.5pt,
        column sep      = 8pt,
        row sep         = 8pt,
        nodes           = {
            minimum width  = 20pt,
            minimum height = 10pt,
            anchor         = center,
            inner sep      = 0pt,
            outer sep      = 0pt,
            scale          = \myscale,
            transform shape
        }
    }
}
\pgfkeys{tikz/mymatrixbrace/.style={decorate,thick}}

\newcommand*\mymatrixbraceright[4][m]{
    \draw[mymatrixbrace] (#1.west|-#1-#3-1.south west) -- node[left=2pt] {#4} (#1.west|-#1-#2-1.north west);
}
\newcommand*\mymatrixbraceleft[4][m]{
    \draw[mymatrixbrace] (#1.east|-#1-#2-1.north east) -- node[right=2pt] {#4} (#1.east|-#1-#2-1.south east);
}
\newcommand*\mymatrixbracetop[4][m]{
    \draw[mymatrixbrace] (#1.north-|#1-1-#2.north west) -- node[above=2pt] {#4} (#1.north-|#1-1-#3.north east);
}
\newcommand*\mymatrixbracebottom[4][m]{
    \draw[mymatrixbrace] (#1.south-|#1-1-#2.north east) -- node[below=2pt] {#4} (#1.south-|#1-1-#3.north west);
}

\tikzset{
    greenish/.style = {
        fill         = green!50!lime!60,
        draw opacity = 0.4,
        draw         = green!50!lime!60,
        fill opacity = 0.1,
    },
    cyanish/.style = {
        fill         = cyan!90!blue!60,
        draw opacity = 0.4,
        draw         = blue!70!cyan!30,
        fill opacity = 0.1,
    },
    orangeish/.style = {
        fill         = orange!90,
        draw opacity = 0.8,
        draw         = orange!90,
        fill opacity = 0.3,
    },
    brownish/.style = {
        fill         = brown!70!orange!40,
        draw opacity = 0.4,
        draw         = brown,
        fill opacity = 0.3,
    },
    purpleish/.style = {
        fill         = violet!90!pink!20,
        draw opacity = 0.5,
        draw         = violet,
        fill opacity = 0.3,    
    }
}


\begin{equation}
    \mathbf{X} = 
    \begin{tikzpicture}[
        baseline={-0.5ex},
        mymatrixenv
    ]
        \matrix [mymatrix,inner sep=4pt](m){
            v_{1,1}
                &v_{1,2}
                &v_{1,3}
                &\textcolor{white}{v_{1,4}}
                &
                &\textcolor{white}{v_{1,6}}
            \\
            v_{2,1}
                &v_{2,2}
                &v_{2,3}
                &
                &
                &
            \\
            v_{3,1}
                &v_{3,2}
                &v_{3,3}
                &v_{3,4}
                &
                &
            \\
            \textcolor{white}{v_{4,1}}
                &
                &v_{4,3}
                &v_{4,4}
                &v_{4,5}
                &v_{4,6}
            \\
                &
                &
                &v_{5,4}
                &v_{5,5}
                &v_{5,6}
            \\
            \textcolor{white}{v_{6,1}}
                &
                &
                &v_{6,4}
                &v_{6,5}
                &v_{6,6}
            \\    
        };
        
        % Background colors
        \begin{scope}[on background layer,rounded corners]
            \node [fit=(m-1-1) (m-3-3), greenish, inner xsep=1.5pt, inner ysep=2.5pt]{};
            \node [fit=(m-1-3) (m-4-3), purpleish, inner xsep=0.5pt, inner ysep=3.5pt]{};
            \node [fit=(m-3-1) (m-3-4), brownish, inner xsep=0.5pt, inner ysep=1.5pt]{};
            \node [fit=(m-3-3) (m-4-4), orangeish]{};
            \node [fit=(m-3-4) (m-6-4), purpleish, inner xsep=0.5pt, inner ysep=3.5pt, yshift=1pt]{};
            \node [fit=(m-4-3) (m-4-6), brownish]{};
            \node [fit=(m-4-4) (m-6-6), cyanish, inner xsep=1.5pt, inner ysep=0.5pt, xshift=-1pt]{};
        \end{scope}

        % Braces     
        \begin{scope}[
            every node/.append style = {
                scale = \myscale,
                transform shape
                },
                very thick
        ]
            \mymatrixbraceright{1}{3}{$B'$}
            \mymatrixbraceright{4}{6}{$B''$}
            \mymatrixbracetop{1}{3}{$C'$}
            \mymatrixbracetop{4}{6}{$C''$}
            \mymatrixbracebottom{3}{3}{$F'$}
            \mymatrixbracebottom{4}{4}{$F''$}
            \mymatrixbraceleft{3}{3}{$E'$}
            \mymatrixbraceleft{4}{4}{$E''$}
        \end{scope} 
\end{tikzpicture}
\label{eq:tikzlabel_01}
\end{equation}

%%%%%%%%%%%%%%%%%%%%%%%%%%%%%%%%%%%%%%%%%%%%%%%%%%
%%% TIKZ NODE REFERENCING                      %%%
%%%%%%%%%%%%%%%%%%%%%%%%%%%%%%%%%%%%%%%%%%%%%%%%%%

% Needed librairies:
%  - matrix
%  - fit
%  - shapes.geometric

\colorlet{Klasse A}{red}
\colorlet{Klasse B}{blue}
\colorlet{probe color}{Klasse A}
\colorlet{correct color}{Klasse A}
\colorlet{incorrect color}{Klasse A}
\colorlet{testproben}{gray!20}

\def\vmark{}

\tikzset{
    tight fit/.style={inner sep=0pt, outer sep=0pt},
    probe color/.code={\colorlet{probe color}{#1}},
    correct color/.code={\colorlet{correct color}{#1}},
    incorrect color/.code={\colorlet{incorrect color}{#1}},
    probe/.style args={#1-#2}{%
        outer sep=0pt,
        shape=rectangle,
        probe #1-#2/.try,
        execute at begin node={%
            % Hide the testproben spike in a style rather than
            % clutter up the main code.
            \begin{tikzpicture}[x=2.75pt,y=1.5pt, scale=0.625]
                \path [draw=probe color] plot [smooth] coordinates {(0,0) 
                (1,2) (2,10) (3,1) (4,3) (5,1) (6,4) (7,0)};
            \end{tikzpicture}}  
    },
    vorhesage/.style args={#1-#2}{
        shape=circle,
        draw=correct color,
        text=white,
        font=\bf\small,
        vorhesage #1-#2/.try,
        incorrect vorhesage #1-#2/.try,
        minimum size=0.625cm    
    },
    surrogat/.style={
        shape=regular polygon,
        regular polygon sides=3,
        minimum height=1cm,
        draw
    }
}


\tikzset{
    % The styles applied to specified testproben and (correct) vorhesage
    testproben/.style args={#1-#2}{
        probe #1-#2/.style={
            fill=testproben
        },
        vorhesage #1-#2/.style={
            fill=correct color,
            execute at begin node=\def\vmark{\ding{51}}% A tick
        }
    },
    % The styles applied to specified testproben and (incorrect) vorhesage
    testproben */.style args={#1-#2}{
        probe #1-#2/.style={
            fill=testproben
        },
        incorrect vorhesage #1-#2/.style={
            fill=incorrect color,
            draw=correct color,
            very thick,
            execute at begin node=\def\vmark{\ding{55}}% A cross
        }
    },
    Daten A/.style={
        A/.try,
        probe color=Klasse A,
    },
    Daten B/.style={
        B/.try,
        probe color=Klasse B,
        shift={(5,0)}
    },
    Vorhesage A/.style={
        A/.try,
        correct color=Klasse A,
        incorrect color=Klasse B,
    },
    Vorhesage B/.style={
        B/.try,
        correct color=Klasse B,
        incorrect color=Klasse A,
        shift={(5,0)}
    },      
    iteration 1/.style={
        % Define the testproben (and vorhesagen) for iteration 1
        A/.style={
            testproben={1-4}, % testproben row 1, column 4
            testproben={2-3}, testproben={2-5},
            testproben *={3-1}, testproben={3-2},
        },
        B/.style={
            testproben={1-2}, testproben={1-3},
            testproben *={2-1},
            testproben={3-4}
        },
    },
    iteration 2/.style={
        A/.style={
            testproben={1-1}, testproben={1-4},testproben={1-5},
            testproben={2-2},
            testproben={3-5}
        },
        B/.style={
            testproben={1-2}, testproben={2-1},
            testproben={3-1}, testproben={3-3}
        }
    },
    iteration 3/.style={
        A/.style={
            testproben={1-5}, testproben={1-4},
            testproben={2-2}, testproben={2-2},
            testproben *={3-1}
        },
        B/.style={
            testproben={1-2}, 
            testproben={2-1},
            testproben={3-3}, testproben={3-4}
        },
    }
}

\begin{figure}[H]
\tiny
\begin{tikzpicture}[scale=0.7, >=stealth]

\foreach \itr in {1,2,3}{
    % Install the A and B styles for this iteration.
    % The A and B styles define the testproben for Klasse A
    % and Kasse B.
    \tikzset{iteration \itr/.try}
    \foreach \g in {1,2,3}{
        \ifcase\g
        \or
            % Draw the probes
            \foreach \K/\I/\J in {A/3/5, B/3/4}{%
                % Install the Daten \K style.
                % For Daten Klasse A this sets the probe color to blue
                % For Daten Klasse B this sets the probe color to red
                % and shifts everything along.
                % In both cases the relevant style (A or B) is `executed'
                % defining which probes are testproben.
                \tikzset{Daten \K/.try}
                \foreach \i in {1, ..., \I}{%
                    \foreach \j in {1, ..., \J}{%
                        \node (probe \itr-\K-\i-\j) at (\j, -\itr*3.5-\i) 
                           [probe=\i-\j] {};
                    }
                }
                % Define a node that fits around all the nodes for this
                % particular Daten Klasse.
                \node [tight fit,fit=(probe \itr-\K-1-1) 
                (probe \itr-\K-\I-\J)] 
                    (iteration \itr\space daten \K){};
            }
        \or
            % Now draw the Surrogats (Surrogaten?)
            \foreach \I in {1,2,3}{
                \tikzset{shift=(probe \itr-B-1-4.east)}
                \node (surrogat \itr-\I) at (1.125,1-\I) [surrogat] {};
                \draw [ultra thick, ->] 
                    (surrogat \itr-\I.west) ++(-0.625,0) -- ++(0.5,0);
                \draw [ultra thick, ->] 
                    (surrogat \itr-\I.east) ++(0.125,0) -- ++(0.5,0);
            }
        \or
            % Finally the Forcasts
            %
            % Shift things along a but from the surrogats.
            \tikzset{shift={(surrogat \itr-1)}, shift=(0:0.5)}
            \foreach \K/\I/\J in {A/3/5, B/3/4}{
                % This is the same as for the Daten Klasse,
                % execept tis time the vorhesage proben are drawn.
                \tikzset{Vorhesage \K/.try}
                \foreach \i in {1, ..., \I}{
                    \foreach \j in {1, ..., \J}{
                        \node (vorhesage \itr-\K-\i-\j) at (\j, -\i+1) 
                        [vorhesage=\i-\j] {\vmark};
                    }
                },
                % Draw a node around each set of vorhesage nodes.
                \node [tight fit, 
                   fit=(vorhesage \itr-\K-1-1) (vorhesage \itr-\K-\I-\J)] 
                (iteration \itr\space vorhesage \K){};
            }               
        \fi%
    }
}

% Now add the labels and delimiters.

\foreach \itr in {1,2,3}{
    \node [tight fit, fit={(iteration \itr\space daten A)}, left 
    delimiter=\{, label={[xshift=-0.5cm]left:\itr. 
    Iteration}] {};
}
\foreach \K in {A, B}{

    \node [tight fit, fit={(iteration 1 daten \K)}, above 
    delimiter=\{,label={[yshift=0.5cm]90:Daten Klassen \K}] {};

    \node [tight fit, fit={(iteration 1 vorhesage \K)}, above 
        delimiter=\{,label={[yshift=0.5cm, align=center]90:{Vorhesage 
        \\Proben \K}}] {};
}

\node [fill=testproben, rounded corners=1ex, below=0.25cm, anchor=north west]  
(testprobe)
    at (probe 3-A-3-1.south east){Testprobe};

\draw [very thick, ->, rounded corners=1ex]
    (testprobe.west) -| (probe 3-A-3-1.south);

\node [below=0.25cm, anchor=north west]  
(trainingsprobe)
    at (probe 3-A-3-4.south east){Trainingsprobe};

\draw [very thick, ->, rounded corners=1ex]
    (trainingsprobe.west) -| (probe 3-A-3-4.south);

\node [below=1cm, anchor=north east]  
(surrogatsmodelle)
    at (surrogat 3-3.south west){$i=3\times k=3$ Surrogatsmodelle};

\draw [very thick, ->, rounded corners=1ex]
    (surrogatsmodelle.east) -| (surrogat 3-3.south);

\node [below=0.5cm, align=left, anchor=north west] (eingeordnet)
    at (vorhesage 3-A-3-1.south east){Testprobe aus Klasse \textcolor{Klasse 
    A}{A} wird \\als Klasse \textcolor{Klasse B}{B} eingeordnet};

\draw [very thick, ->, rounded corners=1ex]
    (eingeordnet.west) -| (vorhesage 3-A-3-1.south);

\end{tikzpicture}
\caption[Template: TikZ node referencing]{Caption under figure.}
\label{fig:tikzlabel_06}
\end{figure}

%%%%%%%%%%%%%%%%%%%%%%%%%%%%%%%%%%%%%%%%%%%%%%%%%%
%%% TABLE WITH ALTERNATING ROW COLORS          %%%
%%%%%%%%%%%%%%%%%%%%%%%%%%%%%%%%%%%%%%%%%%%%%%%%%%

% Needed librairies:
%  - matrix

\tikzset{ 
    table/.style = {
        matrix of nodes,
        nodes in empty cells,
        row sep        = -\pgflinewidth,
        column sep     = -\pgflinewidth,
        minimum height = 1.5em,
        text depth     = 0.5ex,
        text height    = 2ex,
        nodes = {
            rectangle,
            draw  = black,
            align = center
        },
        every even row/.style = {
            nodes = {fill=gray!20}
        },
        column 1/.style = {
            nodes = {
                text width = 2em,
                font       = \bfseries
            }
        },
        row 1/.style = {
            nodes = {
                fill = black,
                text = white,
                font = \bfseries
            }
        }
    }
}

\begin{table}[H]
    \caption[Template: TikZ table with alternating row colors]{Caption over table.}
    \label{tab:tikzlabel_02}

    \vspace{-\baselineskip}
    \begin{center}
        \begin{tikzpicture}
            \matrix (first) [table, text width=6em]{
                & Monday & Tuesday & Wednesday & Thursday & Friday\\
                1 & A & B & C & D & E \\
                2 & F & G & H & J & K \\
                3 & A & B & C & D & E \\
                4 & F & G & H & J & K \\
            };
        \end{tikzpicture}
    \end{center}
\end{table}

%%%%%%%%%%%%%%%%%%%%%%%%%%%%%%%%%%%%%%%%%%%%%%%%%%
%%% FANCY TABLE                                %%%
%%%%%%%%%%%%%%%%%%%%%%%%%%%%%%%%%%%%%%%%%%%%%%%%%%

% Needed librairies:
%  - matrix
%  - calc

\pgfdeclarelayer{background}
\pgfsetlayers{background,main}

\begin{table}[H]
    \caption[Template: TikZ fancy table]{Caption over table.}
    \label{tab:tikzlabel_03}

    \vspace{-\baselineskip}
    \begin{center}
        \begin{tikzpicture}
            \matrix (magic) [
                matrix of nodes,
                draw,
                inner sep = 0,
                nodes     = {
                    minimum width  = 3cm,
                    minimum height = 1cm,
                    draw,
                    very thin
                }
            ] {
                |[fill=red!70]|8 & 1 & 6 \\
                3 & |[left color=cyan,right color=orange]| 5 & 7 \\
                4 & 9 & |[text=red,blue]|2 \\
            };
            
            \draw[thick, violet] (magic-2-1.east) to[out=180,in=270,looseness=0.5] (magic-2-1.north) to[out=270,in=0,looseness=0.5] (magic-2-1.west) to[out=0,in=90,looseness=0.5] (magic-2-1.south) to[out=90,in=180,looseness=0.5] (magic-2-1.east);
            
            \draw[rounded corners=2pt, densely dashed, green!50!gray] ($(magic-1-2.center)+(-0.15,-0.25)$) rectangle ($(magic-1-3.center)+(0.15,0.25)$);
        \end{tikzpicture}
    \end{center}
\end{table}