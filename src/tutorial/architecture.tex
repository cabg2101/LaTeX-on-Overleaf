\begin{figure}[H]
    \centering
    \begin{forest}
        for tree = {
            font=\ttfamily,
            grow'=0,
            child anchor=west,
            parent anchor=south,
            anchor=west,
            calign=first,
            edge path={
                \noexpand\path [draw, thick, \forestoption{edge}]
                (!u.south west) +(15pt,0) |- node[circle,fill,inner sep=2pt] {} (.child anchor)\forestoption{edge label};
            },
            before typesetting nodes={
                if n=1
                {insert before={[,phantom]}}
                {}
            },
            fit=band,
            before computing xy={l=30pt},
        }
        [root/
            [latexmkrc]
            [main.tex]
            [cmds/
                [local\_basics.sty]
                [local\_conversions.sty]
                [local\_kinematics.sty]
                [local\_physics.sty]
                [...]
            ]
            [data/
                [...]
            ]
            [defs/
                [...]
            ]
            [src/
                [...]
            ]
            [tools/
                [acronyms.tex]
                [cover.tex]
                [glossary.tex]
                [ref.bib]
                [settings.tex]
                [styles.tex]
                [...]
            ]
            [...]
        ]
    \end{forest}
    \caption[Global architecture]{Global architecture.}
    \label{fig:tutorial/architecture}
\end{figure}

A previous version of the global architecture included a folder for top-level document files, called \texttt{comp/}, and a data folder that used to be named \texttt{imgs/}. This has been deprecated for future versions of this template. If you end up using the deprecated architecture, just know that the \texttt{doc.tex} file in \S~\ref{sec:tutorial/architecture/src} can be found in your \texttt{comp/} folder.