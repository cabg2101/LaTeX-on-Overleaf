\begin{figure}[H]
    \centering
    \begin{forest}
        for tree = {
            font=\ttfamily,
            grow'=0,
            child anchor=west,
            parent anchor=south,
            anchor=west,
            calign=first,
            edge path={
                \noexpand\path [draw, thick, \forestoption{edge}]
                (!u.south west) +(15pt,0) |- node[circle,fill,inner sep=2pt] {} (.child anchor)\forestoption{edge label};
            },
            before typesetting nodes={
                if n=1
                {insert before={[,phantom]}}
                {}
            },
            fit=band,
            before computing xy={l=30pt},
        }
        [src/
            [doc/
                [sec/
                    [subsec/
                        [subsubsec.tex]
                        [...]
                    ]
                    [subsec.tex]
                    [...]
                ]
                [sec.tex]
                [...]
            ]
            [doc.tex]
            [...]
        ]
    \end{forest}
    \caption[Architecture of section files]{Architecture of section files.}
    \label{fig:tutorial/architecture/src}
\end{figure}

This folder contains the \TeX{} files which represent the sections, subsections, and sub-subsections of a document. Every set of sections / subsections / sub-subsections is stored inside a folder that represents a single compiled document (e.g., \texttt{doc/}), which is defined inside a \TeX{} file at the top level of \texttt{src/} (e.g., \texttt{doc.tex}). Every section can have its own content defined in a \TeX{} file, and then every subsection inside that section does the same in a folder with the section name. Same principle applies for the sub-subsections. Top level files represent the compilation file of each document, which defines everything: settings, cover page, content, etc. In a previous version of my template, these top level files were stored in a separate folder on root, named \texttt{comp/}: this was quickly abandoned because I don't find any significant advantage in separating them from the rest of the document structure. As I said in \S~\ref{sec:tutorial/architecture/data}, it will create a hierarchy that seems bloated for small projects, but allows the project to change in scope and size pretty easily.

I suggest finding short handles for your section names, so it's easier to read in the tree browser on Overleaf. If your document has a lot of sections, or a section has a lot of subsections, etc., you can add a numbering scheme in your \TeX{} file names, though now your naming scheme is tied to the document structure and should be modified when the structure is altered.

In this template, we stop at three division levels, but it's possible to adapt it to include more or less levels (more on that in \S~\ref{sec:tutorial/latex/sec} and \S~\ref{sec:tutorial/preamble/toc}).