\begin{figure}[H]
    \centering
    \begin{forest}
        for tree = {
            font=\ttfamily,
            grow'=0,
            child anchor=west,
            parent anchor=south,
            anchor=west,
            calign=first,
            edge path={
                \noexpand\path [draw, thick, \forestoption{edge}]
                (!u.south west) +(15pt,0) |- node[circle,fill,inner sep=2pt] {} (.child anchor)\forestoption{edge label};
            },
            before typesetting nodes={
                if n=1
                {insert before={[,phantom]}}
                {}
            },
            fit=band,
            before computing xy={l=30pt},
        }
        [tools/
            [acronyms.tex]
            [cover.tex]
            [glossary.tex]
            [ref.bib]
            [settings.tex]
            [styles.tex]
            [...]
        ]
    \end{forest}
    \caption[Architecture of tool files]{Architecture of tool files.}
    \label{fig:tutorial/architecture/tools}
\end{figure}

This folder contains settings, databases, and other global properties for the project. Current files are:
\begin{itemize}
    \item \texttt{acronyms.tex} and \texttt{glossary.tex}, which hold the acronym and glossary entries. If the number of entries becomes overwhelming, it could be broken into small files nested in the \texttt{src/} architecture of the document they're used in;
    \item \texttt{ref.bib}, which holds the \gls{source} data. If the number of \glspl{source} becomes overwhelming, it could be broken into small files nested in the \texttt{src/} architecture of the document they're used in;
    \item \texttt{cover.tex}, which represent the global cover page. Different cover pages can be defined either in the compilation file of a document or in its \texttt{src/} hierarchy;
    \item \texttt{styles.tex}, which defines the various page styles that can be used in your documents (more on that in \S~\ref{sec:tutorial/preamble/listing/style}, \S~\ref{sec:tutorial/preamble/headfoot/style}, and \S~\ref{sec:tutorial/preamble/glossary/style});
    \item \texttt{settings.tex}, which represents the global document preamble. It must be included in every document if you want it to compile, \textbf{no exceptions} (see \S~\ref{sec:tutorial/preamble}).
\end{itemize}