Like every programming language, \LaTeX{} reserves some characters for very specific purposes. They can be included in text with either an escape sequence or a special \gls{command}. Although this should not be an issue with \hologo{XeTeX} or Lua\TeX, some \TeX{} engines also have a hard time dealing with non-\gls{ASCII} input, thus we also need to escape some diacritics.

\begingroup
    \setlength{\columnA}{\dimexpr .13\linewidth}
    \setlength{\columnB}{\dimexpr .27\linewidth}
    \setlength{\columnC}{\dimexpr \linewidth-\columnA-\columnB}
    
    \setlength{\columnA}{\columnA-2\tabcolsep-4\vbar/3}
    \setlength{\columnB}{\columnB-2\tabcolsep-4\vbar/3}
    \setlength{\columnC}{\columnC-2\tabcolsep-4\vbar/3}
    
    \begin{longtable}%
        {|\CT{\columnA}|%
          \CT{\columnB}|%
          \LT{\columnC}|%
        }
        \caption[\LaTeX{} reserved characters]{\LaTeX{} reserved characters.}%
        \label{tab:tutorial/latex/char}\\
        
        \hline
        \multicolumn{1}{|\CC{\columnA}|}{\textbf{Character}}
            &\multicolumn{1}{\CC{\columnB}|}{\textbf{Escape Sequence}}
            &\multicolumn{1}{\CC{\columnC}|}{\textbf{Description}}
        \\\hline
        \endfirsthead
        
        \hline
        \multicolumn{1}{|\CC{\columnA}|}{\textbf{Character}}
            &\multicolumn{1}{\CC{\columnB}|}{\textbf{Escape Sequence}}
            &\multicolumn{1}{\CC{\columnC}|}{\textbf{Description}}
        \\\hline
        \endhead
        
        \&
            &\verb"\&"
            &Used for alignment in certain \glspl{environment}.
        \\\hline
        
        \%
            &\verb"\%"
            &Used mostly for comments.
        \\\hline
        
        \$
            &\verb"\$"
            &Used for inline math.
        \\\hline
        
        \#
            &\verb"\#"
            &Used for \glspl{parameter} in \gls{macro} declarations.
        \\\hline
        
        \{
            &\verb"\{"
            &Used for \glsdisp{group}{grouping}.
        \\\hline
        
        \}
            &\verb"\}"
            &Used for \glsdisp{group}{grouping}.
        \\\hline
        
        \_
            &\verb"\_"
            &Used for subscripts in math mode.
        \\\hline
        
        \textasciicircum
            &\verb"\textasciicircum"
            &Used for superscripts in math mode.
        \\\hline
        
        \textasciitilde
            &\verb"\textasciitilde"
            &Used for non-breakable space.
        \\\hline
        
        \textbackslash
            &\verb"\textbackslash"
            &Used to declare \glspl{macro}.
        \\\hline
    \end{longtable}
\endgroup

\begingroup
    \setlength{\columnA}{\dimexpr .13\linewidth}
    \setlength{\columnB}{\dimexpr .27\linewidth}
    \setlength{\columnC}{\dimexpr \linewidth-\columnA-\columnB}
    
    \setlength{\columnA}{\columnA-2\tabcolsep-4\vbar/3}
    \setlength{\columnB}{\columnB-2\tabcolsep-4\vbar/3}
    \setlength{\columnC}{\columnC-2\tabcolsep-4\vbar/3}
    
    \begin{longtable}%
        {|\CT{\columnA}|%
          \CT{\columnB}|%
          \LT{\columnC}|%
        }
        \caption[\LaTeX{} diacritics]{\LaTeX{} diacritics.}%
        \label{tab:tutorial/latex/diacritics}\\
        
        \hline
        \multicolumn{1}{|\CC{\columnA}|}{\textbf{Character}}
            &\multicolumn{1}{\CC{\columnB}|}{\textbf{Escape Sequence}}
            &\multicolumn{1}{\CC{\columnC}|}{\textbf{Description}}
        \\\hline
        \endfirsthead
        
        \hline
        \multicolumn{1}{|\CC{\columnA}|}{\textbf{Character}}
            &\multicolumn{1}{\CC{\columnB}|}{\textbf{Escape Sequence}}
            &\multicolumn{1}{\CC{\columnC}|}{\textbf{Description}}
        \\\hline
        \endhead
        
        \`{o}
            &\texttt{\textbackslash{}`\{o\}}
            &Grave accent.
        \\\hline
        
        \'{o}
            &\texttt{\textbackslash{}'\{o\}}
            &Acute accent.
        \\\hline
        
        \^{o}
            &\texttt{\textbackslash{}\textasciicircum\{o\}}
            &Circumflex accent.
        \\\hline
        
        \"{o}
            &\texttt{\textbackslash{}"\{o\}}
            &Umlaut, trema or dieresis.
        \\\hline
        
        \H{o}
            &\texttt{\textbackslash{}H\{o\}}
            &Hungarian umlaut (double acute accent).
        \\\hline
        
        \~{o}
            &\texttt{\textbackslash{}\textasciitilde\{o\}}
            &Tilde.
        \\\hline
        
        \c{c}
            &\texttt{\textbackslash{}c\{c\}}
            &Cedilla.
        \\\hline
        
        \k{a}
            &\texttt{\textbackslash{}k\{a\}}
            &Ogonek.
        \\\hline
        
        \={o}
            &\texttt{\textbackslash{}=\{o\}}
            &Macron.
        \\\hline
        
        \b{o}
            &\texttt{\textbackslash{}b\{o\}}
            &Underline.
        \\\hline
        
        \.{o}
            &\texttt{\textbackslash{}.\{o\}}
            &Dot over the letter.
        \\\hline
        
        \d{o}
            &\texttt{\textbackslash{}d\{o\}}
            &Dot under the letter.
        \\\hline
        
        \r{a}
            &\texttt{\textbackslash{}r\{a\}}
            &Ring over the letter.
        \\\hline
        
        \u{o}
            &\texttt{\textbackslash{}u\{o\}}
            &Breve over the letter.
        \\\hline
        
        \v{c}
            &\texttt{\textbackslash{}v\{c\}}
            &Caron or háček over the letter.
        \\\hline
        
        \t{oa}
            &\texttt{\textbackslash{}t\{oa\}}
            &Tie over two letters.
        \\\hline
    \end{longtable}
\endgroup

\begingroup
    \setlength{\columnA}{\dimexpr .13\linewidth}
    \setlength{\columnB}{\dimexpr .27\linewidth}
    \setlength{\columnC}{\dimexpr \linewidth-\columnA-\columnB}
    
    \setlength{\columnA}{\columnA-2\tabcolsep-4\vbar/3}
    \setlength{\columnB}{\columnB-2\tabcolsep-4\vbar/3}
    \setlength{\columnC}{\columnC-2\tabcolsep-4\vbar/3}
    
    \begin{longtable}%
        {|\CT{\columnA}|%
          \CT{\columnB}|%
          \LT{\columnC}|%
        }
        \caption[\LaTeX{} special letters]{\LaTeX{} special letters.}%
        \label{tab:tutorial/latex/letters}\\
        
        \hline
        \multicolumn{1}{|\CC{\columnA}|}{\textbf{Character}}
            &\multicolumn{1}{\CC{\columnB}|}{\textbf{Escape Sequence}}
            &\multicolumn{1}{\CC{\columnC}|}{\textbf{Description}}
        \\\hline
        \endfirsthead
        
        \hline
        \multicolumn{1}{|\CC{\columnA}|}{\textbf{Character}}
            &\multicolumn{1}{\CC{\columnB}|}{\textbf{Escape Sequence}}
            &\multicolumn{1}{\CC{\columnC}|}{\textbf{Description}}
        \\\hline
        \endhead
        
        \oe{} $\left.\middle/\right.$ \OE
            &\texttt{\textbackslash{}oe\{\}} $\left.\middle/\right.$ \texttt{\textbackslash{}OE\{\}}
            &French ligature \guil{oe}.
        \\\hline
        
        \ae{} $\left.\middle/\right.$ \AE
            &\texttt{\textbackslash{}ae\{\}} $\left.\middle/\right.$ \texttt{\textbackslash{}AE\{\}}
            &Latin or scandinavian \guil{ae}.
        \\\hline
        
        \aa{} $\left.\middle/\right.$ \AA
            &\texttt{\textbackslash{}aa\{\}} $\left.\middle/\right.$ \texttt{\textbackslash{}AA\{\}}
            &Scandinavian \guil{a} with circle.
        \\\hline
        
        \l{} $\left.\middle/\right.$ \L
            &\texttt{\textbackslash{}l\{\}} $\left.\middle/\right.$ \texttt{\textbackslash{}L\{\}}
            &Barred \guil{l}.
        \\\hline
        
        \o{} $\left.\middle/\right.$ \O
            &\verb"\o{}" $\left.\middle/\right.$ \verb"\O{}"
            &Slashed \guil{o}.
        \\\hline
        
        \i
            &\texttt{\textbackslash{}i\{\}}
            &Dotless \guil{i}.
        \\\hline
        
        \j
            &\texttt{\textbackslash{}j\{\}}
            &Dotless \guil{j}.
        \\\hline
        
        \ss
            &\texttt{\textbackslash{}ss\{\}}
            &German sharp \guil{s}.
        \\\hline
        
        \S
            &\texttt{\textbackslash{}S\{\}}
            &Section mark.
        \\\hline
        
        \P
            &\texttt{\textbackslash{}P\{\}}
            &Paragraph mark.
        \\\hline
        
        \dag
            &\texttt{\textbackslash{}dag\{\}}
            &Dagger.
        \\\hline
        
        \ddag
            &\texttt{\textbackslash{}ddag\{\}}
            &Double dagger.
        \\\hline
    \end{longtable}
\endgroup