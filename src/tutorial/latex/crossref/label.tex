\glsdisp{label}{Labelling} and \glsdisp{reference}{referencing} is very easy. It only requires two \glspl{command}: \texttt{\textbackslash{}\gls{label}\{\}} and \texttt{\textbackslash{}\glsdisp{reference}{ref}\{\}}. The former creates an \gls{argument} that can be used by the latter to create a \gls{hyperlink} associated with a specific floating \gls{environment} instance, or object, at a specific spot in the document. Most templates (see \S~\ref{sec:tutorial/templates}) add a \gls{label} where most appropriate, except for lists: only ordered lists support \glsdisp{label}{labelling} properly, and their appeal is limited since only items can be \glsdisp{label}{labelled} and not the list itself. Glossary entries, acronyms and \glspl{source} have their own independent, integrated approach: the two \glspl{command} above are unnecessary.

Since the same \glspl{command} can be used to \gls{reference} almost anything, it might get a bit confusing when a lot of \glspl{reference} are introduced. Two mitigation strategies can be used to make it easier to find the right one:
\begin{itemize}
    \item add a few letters to the \gls{label} to describe what you are \glsdisp{reference}{referencing} (see table \ref{tab:tutorial/latex/crossref/label}). It's common practice among \LaTeX{} users. Some \glspl{package}, such as \texttt{fancyref}, rely on this metadata;
    \item add your directory inside the \gls{label}. \Glspl{label} can get quite lengthy with that method, but it's easier to find when looking at the document structure and using the auto-completion feature of Overleaf.
\end{itemize}

Examples of \glspl{label} can be seen in table \ref{tab:tutorial/latex/crossref/label}. Note that, past the colon (:), words like \guil{doc}, \guil{sec}, \guil{subsec}, and other \glsdisp{reference}{refer} to the name of the document, the section, the subsection, etc., inside the project architecture. For instance, if the document folder is named \guil{tutorial}, you can replace \guil{doc} with it in all your \glspl{label} inside that document (more information about the project architecture in \S~\ref{sec:tutorial/architecture}).

\begingroup
    \setlength{\columnA}{\dimexpr \linewidth/8}
    \setlength{\columnB}{\dimexpr \columnA}
    \setlength{\columnC}{\dimexpr \linewidth-\columnA-\columnB}
    
    \setlength{\columnA}{\columnA-2\tabcolsep-4\vbar/3}
    \setlength{\columnB}{\columnB-2\tabcolsep-4\vbar/3}
    \setlength{\columnC}{\columnC-2\tabcolsep-4\vbar/3}
    
    \begin{longtable}%
        {|\CC{\columnA}|%
          \CC{\columnB}|%
          \LC{\columnC}|%
        }
        \caption[\LaTeX{} labelling examples]{\LaTeX{} \glsdisp{label}{labelling} examples.}%
        \label{tab:tutorial/latex/crossref/label}\\
        
        \hline
        \multicolumn{2}{|c|}{\textbf{Type}}
            &\multicolumn{1}{c|}{\textbf{Examples}}
        \\\hline
        \endfirsthead
        
        \hline
        \multicolumn{2}{|c|}{\textbf{Type}}
            &\multicolumn{1}{c|}{\textbf{Examples}}
        \\\hline
        \endhead
        
        Section
            &\texttt{sec:}
            &-- \verb"\label{sec:doc/sec}"
             \newline -- \verb"\label{sec:doc/sec/subsec}"
             \newline -- \verb"\label{sec:doc/sec/subsec/subsubsec}"
             \newline -- \ldots
        \\\hline
        
        Figure
            &\texttt{fig:}
            &-- \verb"\label{fig:doc/sec/fig}"
             \newline -- \verb"\label{fig:doc/sec/subsec/fig/subfig}"
             \newline -- \ldots
        \\\hline
        
        Table
            &\texttt{tab:}
            &-- \verb"\label{tab:doc/sec/tab}"
             \newline -- \verb"\label{tab:doc/sec/subsec/tab/subtab}"
             \newline -- \ldots
        \\\hline
        
        Equation
            &\texttt{eq:}
            &-- \verb"\label{eq:doc/sec/equation}"
             \newline -- \verb"\label{eq:doc/sec/subsec/equation}"
             \newline -- \ldots
        \\\hline
        
        Code listing
            &\texttt{lst:}
            &-- \verb"\label{lst:doc/sec/code}"
             \newline -- \verb"\label{lst:doc/sec/subsec/code}"
             \newline -- \ldots
        \\\hline
    \end{longtable}
\endgroup