Mathematical equations can be included in your document with a variety of \glspl{environment}. Table \ref{tab:tutorial/latex/math/env} lists the common ones.

\textbf{Note}: Some math \glspl{environment} will allow automatic numbering of equations by default, such as \texttt{equation}, \texttt{align}, and \texttt{eqnarray}. To prevent this behavior, you can declare them using an asterisk:
\begin{center}
    \verb"\begin{align*}...\end{align*}"
\end{center}

Maybe you want to number the last line only. In order to do this, you will have to use a numbering \gls{environment} and declare \verb"\nonumber" on every line that shouldn't have a number.

You can also allow \glsdisp{reference}{referencing} of your equations or equation sets by adding a \gls{label} before closing the math \gls{environment} (see \S~\ref{sec:tutorial/latex/crossref/label} for \glsdisp{label}{labelling} scheme and \S~\ref{sec:tutorial/templates/math} for examples).

\begingroup
    \setlength{\columnA}{\dimexpr .25\linewidth}
    \setlength{\columnB}{\dimexpr \linewidth-\columnA}
    
    \setlength{\columnA}{\columnA-2\tabcolsep-3\vbar/2}
    \setlength{\columnB}{\columnB-2\tabcolsep-3\vbar/2}
    
    \begin{longtable}%
        {|\CC{\columnA}|%
          \LC{\columnB}|%
        }
        \caption[\LaTeX{} mathematical \glsentryplural{environment}]{\LaTeX{} mathematical \glspl{environment}.}%
        \label{tab:tutorial/latex/math/env}\\
        
        \hline
        \textbf{\Gls{environment}}
            &\multicolumn{1}{c|}{\textbf{Description}}
        \\\hline
        \endfirsthead
        
        \hline
        \textbf{\Gls{environment}}
            &\multicolumn{1}{c|}{\textbf{Description}}
        \\\hline
        \endhead
        
        {\small \texttt{math}}
            &Inline equations (inside text).
             \newline -- Shortcut: \$\ldots\$
        \\\hline
        
        \texttt{displaymath}
            &Unnumbered equations that stand on their own line.
             \newline -- Shortcut: \textbackslash[\ldots\textbackslash] 
        \\\hline
        
        \texttt{equation}
            &Numbered equations that stand on their own line.
        \\\hline
        
        \texttt{align}
            &Two-column \gls{environment} that can be used to align equations (alignment character: \&). The first column is right-aligned, the second is left-aligned.
        \\\hline
        
        \texttt{eqnarray}
            &Three-column \gls{environment} that can be used to align equations (alignment character: \&). The first column is right-aligned, the second is centered, and the third is left-aligned.
        \\\hline
        
        \texttt{matrix}
            &Matrices (alignment character: {\large\texttt{\&}}; row change: \verb"\\").
             \newline -- \texttt{pmatrix}: matrix with parentheses;
             \newline -- \texttt{bmatrix}: matrix with brackets;
             \newline -- \texttt{Bmatrix}: matrix with braces;
             \newline -- \texttt{vmatrix}: matrix with vertical bars;
             \newline -- \texttt{Vmatrix}: matrix with double vertical bars.
        \\\hline
    \end{longtable}
\endgroup