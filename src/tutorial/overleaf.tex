\textit{This project was implemented in Overleaf. It is not necessary to understand the section, although it provides examples of what is discussed. For proper illustration of my words, start a project with your own Overleaf account and follow along, or ask for viewing privileges on this project if you know me personally.}

\hr

If you go over the Overleaf menu of this project, most settings will be straightforward to you, especially those that have already programmed and are familiar working with an \gls{IDE}. You will see settings such as \guil{spell check}, \guil{code check}, \guil{auto-completion}, etc. They are pretty self-explanatory. All fine and dandy, until you get to \guil{compiler} and \guil{\TeX{} Live version}:
\begin{itemize}
    \item What's \TeX{} Live?
    \item What \TeX{} Live version should I use? Always the most up-to-date?
    \item What compiler should I choose for my project?
    \item What distinguishes the different compilers?
\end{itemize} 

Fear not. Most of the different options can be found in table \ref{tab:tutorial/overleaf/tex}, and then some. Each level will be described in more details in the following subsections. More information is also available on Overleaf \parencite{web:overleaf-flavours-tex}. For simplicity, \guil{\LaTeX} may be used pretty loosely in the rest of the document to refer to the typesetting system in part or in its entirety.

\begingroup
    \setlength{\columnA}{\dimexpr .16\linewidth}
    \setlength{\columnB}{\dimexpr .16\linewidth}
    \setlength{\columnC}{\dimexpr \linewidth-\columnA-\columnB}
    
    \setlength{\columnA}{\columnA-2\tabcolsep-4\vbar/3}
    \setlength{\columnB}{\columnB-2\tabcolsep-4\vbar/3}
    \setlength{\columnC}{\columnC-2\tabcolsep-4\vbar/3}
    
    \begin{longtable}%
        {|\CT{\columnA}|%
          \LT{\columnB}|%
          \LT{\columnC}|%
        }
        \caption[\TeX{} architecture]{\TeX{} architecture.}%
        \label{tab:tutorial/overleaf/tex}\\
        
        \hline
        \multicolumn{1}{|\CC{\columnA}|}{\textbf{Level}}
            &\multicolumn{1}{\CC{\columnB}|}{\textbf{Examples}}
            &\multicolumn{1}{\CC{\columnC}|}{\textbf{Description}}
        \\\hline
        \endfirsthead
        
        \hline
        \multicolumn{1}{|\CC{\columnA}|}{\textbf{Level}}
            &\multicolumn{1}{\CC{\columnB}|}{\textbf{Examples}}
            &\multicolumn{1}{\CC{\columnC}|}{\textbf{Description}}
        \\\hline
        \endhead
        
        Engine
            &-- \TeX
             \newline -- \glsdisp{PDF}{pdf}\TeX
             \newline -- \hologo{XeTeX}
             \newline -- Lua\TeX
             \newline -- $\cdots$
            &These are the executable binaries which implement different \TeX{} variants, known as the \guil{compiler} option in Overleaf.
        \\\hline
        
        Format
            &-- \TeX
             \newline -- \LaTeX
             \newline -- Op\TeX
             \newline -- $\cdots$
            &These are the \TeX-based languages in which one actually writes documents. When someone says \guil{\TeX{} is giving me a mysterious error}, they usually mean a format. If no special format is defined, one can write in plain \TeX.
        \\\hline
        
        Distribution
            &-- \TeX{} Live
             \newline -- MiK\TeX
             \newline -- $\cdots$
            &These are the large, coherent collections of \TeX-related software to be downloaded and installed. When someone says \guil{I need to install \TeX{} on my machine}, they're usually looking for a distribution.
        \\\hline
        
        Editor
            &-- Overleaf
             \newline -- \hologo{LyX}
             \newline -- Vim
             \newline -- $\cdots$
            &These editors are what you use to create or modify a document file.
        \\\hline
    \end{longtable}
\endgroup