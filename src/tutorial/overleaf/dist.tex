Distributions can regroup large collections of \glspl{macro}, \glspl{package}, and \glspl{class} into a single installer. The most common ones are MiK\TeX{} and \TeX{} Live. More and more people are leaning towards \TeX{} Live over time, as it is maintained by a community of users instead of a single person or a small group: it can stand the test of time more easily. MiK\TeX{} can offer some features that are not available in the other distribution or are better implemented, so the distribution you select for a local install is entirely up to you and your depth of knowledge. Overleaf uses \TeX{} Live on their servers, so we don't have a choice when writing documents online.

\TeX{} Live is intended to be a straightforward way to get up and running with the \TeX{} document production system. It provides a comprehensive \TeX{} system with binaries for most flavors of Unix, including GNU/Linux, macOS, and also Windows. It includes all the major \TeX{}-related programs, \glspl{package}, fonts that are \gls{FOSS}, and support for many languages \parencite{web:tex-live}. It has been in development since 1996, and a new version comes out every year, integrating new features and bug fixes. \TeX{} in general moves slowly, so you can stay on whatever version of \TeX{} Live your document was created with---unless you need the updated utilities of a newer version. The present project was tested with \TeX{} Live 2020.