If you're writing a document locally, you can use any editor of your choice: \hologo{LyX}, Emacs, Vim, Notepad++\ldots{} Being bound to the realm of the World Wide Web, and in our cozy, no-\gls{command-prog}-line world, we will stick with Overleaf.

If you hear the name \guil{Share\LaTeX} being thrown around, they merged with Overleaf a couple years ago, and their documentation and accounts carried over. There was also \guil{Write\LaTeX}, although I have no information about that one---I suspect it was Overleaf's previous name.

\noindent The basics of an Overleaf project include:
\begin{itemize}
    \item theming, to customize source code colors and syntax highlighting the way you like it;
    \item choice of engine and, although limited, distribution version;
    \item synchronization with Dropbox, Git or GitHub;
    \item code auto-completion;
    \item code / text review;
    \item multi-user editing and sharing;
    \item gallery publishing;
    \item history logging;
    \item in-project messaging;
    \item rich text mode.
\end{itemize}