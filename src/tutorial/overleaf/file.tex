Every step in the document compilation chain can use or generate files: the engine, the format, the \glspl{package}, etc. Most types of input and output files can be found in tables \ref{tab:tutorial/overleaf/file/input} and \ref{tab:tutorial/overleaf/file/output}.

Most Overleaf users will only need to worry about the highlighted files. As you get on your way to become a \TeX{}nician, or if you start running your projects locally, you may start looking into more and more file types (see \S~\ref{sec:tutorial/texnician}).

\begingroup
    \setlength{\columnA}{\dimexpr .16\linewidth}
    \setlength{\columnB}{\dimexpr .16\linewidth}
    \setlength{\columnC}{\dimexpr \linewidth-\columnA-\columnB}
    
    \setlength{\columnA}{\columnA-2\tabcolsep-4\vbar/3}
    \setlength{\columnB}{\columnB-2\tabcolsep-4\vbar/3}
    \setlength{\columnC}{\columnC-2\tabcolsep-4\vbar/3}
    
    \begin{longtable}%
        {|\CC{\columnA}|%
          \CC{\columnB}|%
          \LC{\columnC}|%
        }
        \caption[\TeX{} input file types]{\TeX{} input file types.}%
        \label{tab:tutorial/overleaf/file/input}\\
        
        \hline
        \textbf{File}
            &\textbf{Extension}
            &\multicolumn{1}{c|}{\textbf{Description}}
        \\\hline
        \endfirsthead
        
        \hline
        \textbf{File}
            &\textbf{Extension}
            &\multicolumn{1}{c|}{\textbf{Description}}
        \\\hline
        \endhead
        
        \rowcolor{gray!20} Source
            &\texttt{.tex}
            &General purpose file. It may include text, symbols, mathematical expressions, graphics, and more.
        \\\hline
        
        Class
            &\texttt{.cls}
            &File that stores a document \gls{class} with predefined typeset configuration. It can be used to create custom articles, books, posters, etc.
        \\\hline
        
        Class options
            &\texttt{.clo}
            &File that stores a document \gls{class}' options. Usually, they are defined in a \texttt{.cls} file, but they are occasionally defined separately.
        \\\hline
        
        Style
            &\texttt{.sty}
            &File that contains \glspl{macro} of specific styling attributes.
        \\\hline
        
        Definition
            &\texttt{.def}
            &File that contains lists of \glspl{command} and \glspl{definition} that would clog a \texttt{.sty} file.
        \\\hline
        
        \rowcolor{gray!20} \Glspl{source}
            &\texttt{.bib}
            &Database that stores \glspl{source} for the bibliography. 
        \\\hline
        
        Bib\TeX{} style
            &\texttt{.bst}
            &File that contains styling attributes for the bibliography.
        \\\hline
        
        Bibliography style
            &\texttt{.bbx}
            &File that contains styling attributes for the bibliography generated by the \texttt{biblatex} \gls{package}.
        \\\hline
        
        Cite style
            &\texttt{.cbx}
            &File that contains styling attributes for \glspl{citation} generated by the \texttt{biblatex} \gls{package}.
        \\\hline
        
        Documented source
            &\texttt{.dtx}
            &File that contains both content and documentation. Useful to generate a \LaTeX{} \gls{package} with its associated documentation.
        \\\hline
        
        Installation
            &\texttt{.ins}
            &File that stores instructions to extract template files out of a \texttt{.dtx} file.
        \\\hline
        
        Configuration
            &\texttt{latexmkrc}
            &File that stores \gls{command-prog} line \glspl{argument} for automatic document compilation.
        \\\hline
    \end{longtable}
\endgroup

\begingroup
    \setlength{\columnA}{\dimexpr .16\linewidth}
    \setlength{\columnB}{\dimexpr .16\linewidth}
    \setlength{\columnC}{\dimexpr \linewidth-\columnA-\columnB}
    
    \setlength{\columnA}{\columnA-2\tabcolsep-4\vbar/3}
    \setlength{\columnB}{\columnB-2\tabcolsep-4\vbar/3}
    \setlength{\columnC}{\columnC-2\tabcolsep-4\vbar/3}
    
    \begin{longtable}%
        {|\CC{\columnA}|%
          \CC{\columnB}|%
          \LC{\columnC}|%
        }
        \caption[\TeX{} output file types]{\TeX{} output file types.}%
        \label{tab:tutorial/overleaf/file/output}\\
        
        \hline
        \textbf{File}
            &\textbf{Extension}
            &\multicolumn{1}{c|}{\textbf{Description}}
        \\\hline
        \endfirsthead
        
        \hline
        \textbf{File}
            &\textbf{Extension}
            &\multicolumn{1}{c|}{\textbf{Description}}
        \\\hline
        \endhead
        
        \Gls{DVI}
            &\texttt{.\glsdisp{DVI}{dvi}}
            &The original output file for documents when \TeX{} was created. They consist of binary data describing the visual layout of a document in a manner not reliant on any specific image format, display hardware or printer.
        \\\hline
        
        \rowcolor{gray!20} \Gls{PDF}
            &\texttt{.\glsdisp{PDF}{pdf}}
            &The default document output file for \glsdisp{PDF}{pdf}\TeX, \hologo{XeTeX}, Lua\TeX. For all practical purposes, you should always compile with this type of output.
        \\\hline
        
        Auxiliary
            &\texttt{.aux}
            &File that saves information for the implementation of \glspl{reference}, footnotes, and bibliographies, among other things.
        \\\hline
        
        Log
            &\texttt{.log}
            &File that stores all messages of the compilation, like errors and warnings.
        \\\hline
        
        Font definition
            &\texttt{.fd}
            &File that contains the font information used to generate the document output.
        \\\hline
        
        \hologo{BibTeX}
            &\texttt{.bbl}
            &\hologo{BibTeX} output for insertion into the document output.
        \\\hline
        
        \hologo{BibTeX} log
            &\texttt{.blg}
            &File that contains compilation messages specific to \hologo{BibTeX}.
        \\\hline
        
        Bibliography control
            &\texttt{.bcf}
            &Control file for the \texttt{biblatex} \gls{package}.
        \\\hline
        
        Bibliography \glsname{XML}
            &\texttt{.run.\glsdisp{XML}{xml}}
            &\Glsfirst{XML} file for the bibliography engine biber.
        \\\hline
        
        Table of contents
            &\texttt{.\glsdisp{ToC}{toc}}
            &Auxiliary file that contains the \gls{ToC}.
        \\\hline
        
        List of figures
            &\texttt{.\glsdisp{LoF}{lof}}
            &Auxiliary file that contains the \gls{LoF}.
        \\\hline
        
        List of tables
            &\texttt{.\glsdisp{LoT}{lot}}
            &Auxiliary file that contains the \gls{LoT}.
        \\\hline
        
        Glossary
            &\texttt{.glo}
            &Glossary file used to generate the document output.
        \\\hline
        
        Glossary log
            &\texttt{.glg}
            &File that contains compilation messages specific to glossaries.
        \\\hline
    \end{longtable}
\endgroup