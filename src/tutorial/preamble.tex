As you read along the past sections, you might have thought \guil{this guide seems pretty well made, \LaTeX{} must be magical!} If you \textit{really} read along, you might have thought instead \guil{this guide seems pretty well made... Where's the real work hidden?} If you didn't think of any sentence resembling these two, that means I need to work on my prediction skills. Still, to answer the second one: this is where the real work is hidden. The preamble, which encompasses everything that comes before the start of the document, allows us to define the style, the \glspl{package}, the \glspl{command} we will use later in writing the document. It can be created in a modular fashion with many different files that play together for given combinations, such as \gls{class}, style, \gls{definition}, and section files (see \S~\ref{sec:tutorial/overleaf/file}). I have to admit that my preamble setup is not as optimized as I would like it to be since:
\begin{itemize}
    \item I'm keeping everything in a single file because I'm constantly tweaking it, and managing feature changes in multiple files would get tiresome;
    \item I haven't had the chance to work on document \glspl{class} other than articles, thus my focus has not been on breaking down \gls{class} and style options;
    \item bugs can appear depending on the load order of \glspl{package}, which is really annoying and hurts flexibility.
\end{itemize}

Consequently, almost everything is contained within the \texttt{settings.tex} file, except for some custom \glspl{command} and the \texttt{glossaries} \gls{package} which doesn't play nice when not called within the main document file. The following sections will detail the various considerations that went into making this guide, as well as more obscure \glspl{macro} that can help you understand better the inner workings.