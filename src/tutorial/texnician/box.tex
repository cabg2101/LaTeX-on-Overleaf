\vfill
\begin{center}
    \Huge \MakeUppercase{Add rules from \TeX{}book}
\end{center}
\vfill

Font dimensions are font parameters used by \TeX{} to build the bounding boxes of each and every element in the document. For example, you can think of an equation as a big box broken down into smaller and smaller boxes until you're defining the size for each individual character. \TeX{} uses rules to calculate these boxes, and most parameters must be supplied by the font. There are parameters for text (first family, or \texttt{\textbackslash{}fam1}), for math symbols (\texttt{\textbackslash{}fam2}), and for extension fonts (\texttt{\textbackslash{}fam3}). Each parameter can be found in tables \ref{tab:tutorial/latex/dim/font/text} to \ref{tab:tutorial/latex/dim/font/extension}. More details about the rules can be found in \citename{book:knuth-tex-metafont}{author}'s \citefield{book:knuth-typesetting-a}{title} \parencite{book:knuth-typesetting-a}.

\begingroup
    \setlength{\columnA}{\dimexpr .25\linewidth}
    \setlength{\columnB}{\dimexpr .25\linewidth}
    \setlength{\columnC}{\dimexpr \linewidth-\columnA-\columnB}
    
    \setlength{\columnA}{\columnA-2\tabcolsep-4\vbar/3}
    \setlength{\columnB}{\columnB-2\tabcolsep-4\vbar/3}
    \setlength{\columnC}{\columnC-2\tabcolsep-4\vbar/3}
    
    \begin{longtable}%
        {|\CC{\columnA}|%
          \CC{\columnB}|%
          \LC{\columnC}|%
        }
        \caption[\LaTeX{} text font dimensions]{\LaTeX{} text font dimensions.}%
        \label{tab:tutorial/latex/dim/font/text}\\
        
        \hline
        \multicolumn{1}{|\CC{\columnA}|}{\textbf{Command}}
            &\multicolumn{1}{\CC{\columnB}|}{\textbf{Name}}
            &\multicolumn{1}{\CC{\columnC}|}{\textbf{Description}}
        \\\hline
        \endfirsthead
        
        \hline
        \multicolumn{1}{|\CC{\columnA}|}{\textbf{Command}}
            &\multicolumn{1}{\CC{\columnB}|}{\textbf{Name}}
            &\multicolumn{1}{\CC{\columnC}|}{\textbf{Description}}
        \\\hline
        \endhead
        
        \verb"\fontdimen1"
            &Slant per point
            &Used for italic correction.
        \\\hline
        
        \verb"\fontdimen2"
            &Interword space
            &Standard width of the control space.
        \\\hline
        
        \verb"\fontdimen3"
            &Interword stretch
            &Amount of stretch applied to the control space.
        \\\hline
        
        \verb"\fontdimen4"
            &Interword shrink
            &Amount of shrink applied to the control space.
        \\\hline
        
        \verb"\fontdimen5"
            &\guil{x} height
            &Used to determine the ex unit.
        \\\hline
        
        \verb"\fontdimen6"
            &Quad width
            &Used to determine the em unit.
        \\\hline
        
        \verb"\fontdimen7"
            &Extra space
            &Additional space after sentence punctuation.
        \\\hline
    \end{longtable}
\endgroup

\begingroup
    \setlength{\columnA}{\dimexpr .25\linewidth}
    \setlength{\columnB}{\dimexpr .25\linewidth}
    \setlength{\columnC}{\dimexpr \linewidth-\columnA-\columnB}
    
    \setlength{\columnA}{\columnA-2\tabcolsep-4\vbar/3}
    \setlength{\columnB}{\columnB-2\tabcolsep-4\vbar/3}
    \setlength{\columnC}{\columnC-2\tabcolsep-4\vbar/3}
    
    \begin{longtable}%
        {|\CC{\columnA}|%
          \CC{\columnB}|%
          \LC{\columnC}|%
        }
        \caption[\LaTeX{} math symbols font dimensions]{\LaTeX{} math symbols font dimensions.}%
        \label{tab:tutorial/latex/dim/font/math}\\
        
        \hline
        \multicolumn{1}{|\CC{\columnA}|}{\textbf{Command}}
            &\multicolumn{1}{\CC{\columnB}|}{\textbf{Name}}
            &\multicolumn{1}{\CC{\columnC}|}{\textbf{Description}}
        \\\hline
        \endfirsthead
        
        \hline
        \multicolumn{1}{|\CC{\columnA}|}{\textbf{Command}}
            &\multicolumn{1}{\CC{\columnB}|}{\textbf{Name}}
            &\multicolumn{1}{\CC{\columnC}|}{\textbf{Description}}
        \\\hline
        \endhead
        
        \verb"\fontdimens1"
            &\multicolumn{2}{\CC{\columnB+\columnC+2\tabcolsep+\vbar}|}{
                \multirow{4}{\linewidth}{
                    \centering
                    Same as text font dimensions (see table \ref{tab:tutorial/latex/dim/font/text}).
                }
            }
        \\\cline{1-1}
        
        \verb"\fontdimens2"
            &\multicolumn{2}{\CC{\columnB+\columnC+2\tabcolsep+\vbar}|}{}
        \\\cline{1-1}
        
        \verb"\fontdimens3"
            &\multicolumn{2}{\CC{\columnB+\columnC+2\tabcolsep+\vbar}|}{}
        \\\cline{1-1}
        
        \verb"\fontdimens4"
            &\multicolumn{2}{\CC{\columnB+\columnC+2\tabcolsep+\vbar}|}{}
        \\\hline
        
        \verb"\fontdimens5"
            &\guil{x} height
            &Used to determine the math ex unit.
        \\\hline
        
        \verb"\fontdimens6"
            &Quad width
            &Used to determine the math em unit.
        \\\hline
        
        \verb"\fontdimens7"
            &\multicolumn{2}{\CC{\columnB+\columnC+2\tabcolsep+\vbar}|}{Same as text font dimensions (see table \ref{tab:tutorial/latex/dim/font/text}).}
        \\\hline
        
        \verb"\fontdimens8"
            &Numerator \#1
            &Standard size numerator shift.
        \\\hline
        
        \verb"\fontdimens9"
            &Numerator \#2
            &Alternative numerator shift.
        \\\hline
        
        \verb"\fontdimens10"
            &Numerator \#3
            &Alternative numerator shift.
        \\\hline
        
        \verb"\fontdimens11"
            &Denominator \#1
            &Standard size denominator shift.
        \\\hline
        
        \verb"\fontdimens12"
            &Denominator \#2
            &Alternative denominator shift.
        \\\hline
        
        \verb"\fontdimens13"
            &Superscript \#1
            &Standard size superscript position.
        \\\hline
        
        \verb"\fontdimens14"
            &Superscript \#2
            &Alternative superscript position.
        \\\hline
        
        \verb"\fontdimens15"
            &Superscript \#3
            &Alternative superscript position.
        \\\hline
        
        \verb"\fontdimens16"
            &Subscript \#1
            &Standard size subscript position.
        \\\hline
        
        \verb"\fontdimens17"
            &Subscript \#2
            &Alternative subscript position.
        \\\hline
        
        \verb"\fontdimens18"
            &Superscript drop
            &Standard superscript shift.
        \\\hline
        
        \verb"\fontdimens19"
            &Subscript drop
            &Standard subscript shift.
        \\\hline
        
        \verb"\fontdimens20"
            &Delimiter \#1
            &Minimum display math delimiter size.
        \\\hline
        
        \verb"\fontdimens21"
            &Delimiter \#2
            &Minimum text math delimiter size.
        \\\hline
        
        \verb"\fontdimens22"
            &Axis height
            &Height of operator~/~delimiter center axis.
        \\\hline
    \end{longtable}
\endgroup

\begingroup
    \setlength{\columnA}{\dimexpr .25\linewidth}
    \setlength{\columnB}{\dimexpr .25\linewidth}
    \setlength{\columnC}{\dimexpr \linewidth-\columnA-\columnB}
    
    \setlength{\columnA}{\columnA-2\tabcolsep-4\vbar/3}
    \setlength{\columnB}{\columnB-2\tabcolsep-4\vbar/3}
    \setlength{\columnC}{\columnC-2\tabcolsep-4\vbar/3}
    
    \begin{longtable}%
        {|\CC{\columnA}|%
          \CC{\columnB}|%
          \LC{\columnC}|%
        }
        \caption[\LaTeX{} extension font dimensions]{\LaTeX{} extension font dimensions.}%
        \label{tab:tutorial/latex/dim/font/extension}\\
        
        \hline
        \multicolumn{1}{|\CC{\columnA}|}{\textbf{Command}}
            &\multicolumn{1}{\CC{\columnB}|}{\textbf{Name}}
            &\multicolumn{1}{\CC{\columnC}|}{\textbf{Description}}
        \\\hline
        \endfirsthead
        
        \hline
        \multicolumn{1}{|\CC{\columnA}|}{\textbf{Command}}
            &\multicolumn{1}{\CC{\columnB}|}{\textbf{Name}}
            &\multicolumn{1}{\CC{\columnC}|}{\textbf{Description}}
        \\\hline
        \endhead
        
        \verb"\fontdimenx1"
            &\multicolumn{2}{\CC{\columnB+\columnC+2\tabcolsep+\vbar}|}{
                \multirow{7}{\linewidth}{
                    \centering
                    Same as text font dimensions (see table \ref{tab:tutorial/latex/dim/font/text}).
                }
            }
        \\\cline{1-1}
        
        \verb"\fontdimenx2"
            &\multicolumn{2}{\CC{\columnB+\columnC+2\tabcolsep+\vbar}|}{}
        \\\cline{1-1}
        
        \verb"\fontdimenx3"
            &\multicolumn{2}{\CC{\columnB+\columnC+2\tabcolsep+\vbar}|}{}
        \\\cline{1-1}
        
        \verb"\fontdimenx4"
            &\multicolumn{2}{\CC{\columnB+\columnC+2\tabcolsep+\vbar}|}{}
        \\\cline{1-1}
        
        \verb"\fontdimenx5"
            &\multicolumn{2}{\CC{\columnB+\columnC+2\tabcolsep+\vbar}|}{}
        \\\cline{1-1}
        
        \verb"\fontdimenx6"
            &\multicolumn{2}{\CC{\columnB+\columnC+2\tabcolsep+\vbar}|}{}
        \\\cline{1-1}
        
        \verb"\fontdimenx7"
            &\multicolumn{2}{\CC{\columnB+\columnC+2\tabcolsep+\vbar}|}{}
        \\\hline
        
        \verb"\fontdimenx8"
            &Default rule thickness
            &-- Default fraction line;
             \newline -- Minimum subscript / superscript separation.
        \\\hline
        
        \verb"\fontdimenx9"
            &Operator spacing \#1
            &Used to control limits spacing.
        \\\hline
        
        \verb"\fontdimenx10"
            &Operator spacing \#2
            &Alternative limits spacing.
        \\\hline
        
        \verb"\fontdimenx11"
            &Operator spacing \#3
            &Alternative limits spacing.
        \\\hline
        
        \verb"\fontdimenx12"
            &Operator spacing \#4
            &Alternative limits spacing.
        \\\hline
        
        \verb"\fontdimenx13"
            &Operator spacing \#5
            &Alternative limits spacing.
        \\\hline
    \end{longtable}
\endgroup